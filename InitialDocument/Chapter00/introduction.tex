%-----------------------------------------------------------------------------------------

\section{Introduction}
%-----------------------------------------------------------------------------------------

\subsection{Motivation and Context}
Machine Learning algorithms have become ubiquitous in modern life. Powering
social media feeds, email spam filters, advertising personalisation and even
identifying breast cancer more accurately and earlier than doctors.
\cite{Mammograms2020} To train these Machine Learning algorithms large datasets
are needed. The more nuanced and complex the problem being solved the more data
is necessary. As the scale of problems we are trying to solve dramatically
increase, the scale of datasets are becoming truly gargantuan. Since 2008 Google
has been processing more 20PB of data a day using their MapReduce algorithm.
\cite{googlemapreduce2008} While services like the Internet Archive as of 2020
contain over 70PB in its database. We now have labeled datasets such as AViD
have 467k videos and 887 action classes, which is in the order of terabytes.
\cite{piergiovanni2020avid} 
Whilst the data grows the as does the Machine Learning models in order to to
obtain ever more accurate results. The cutting edge GTP-3 Natural Language
Processing model contains 175 Billion parameters. \cite{fewshowlearners2020gpt}
And efforts are being made to create models with trillions of parameters.
\cite{rajbhandari2020zero}
\par

Deriving meaning from these vast quantities of data to obtaining nuanced
insights from them is a difficult task. Not only because deeper insights into
data require a larger Machine Learning models. But because more data is needed
to populate the parameters of these models. Both of these factors contribute to
the need to distribute the computation of the model across multiple nodes
otherwise known as Distributed Machine Learning. Distributed Machine Learning is
often a pre-requisite for training models now datasets and models are becoming
so large. \cite{ParameterServers}
\par

% is becoming more complex as the insights we seek to solve are increasingly nuanced. It
% therefore follows that out Machine Learning models and algorithms are becoming
% ever more complex, requiring more data to be trained sufficiently. Due to this,
% distributed machine learning has become a pre-requisite for training sizeable
% models with an equally sizeable dataset.

% revise this
The popular current solution is to have multiple machines compute the model
together, communicating the improvements that they've made to each other. The
model goes from operating on a single machine possessing all the data and
needing to do all the computation, to a worker and parameter server paradigm. In
which the parameter server contains the model and the workers perform operations
on it using test data segmented between them. \cite{LI2014ParameterServers}
\par

There are two main variations with respect to the operation of the workers in
parameter server model: 1) The parameter server has to wait for the last worker
to be finished before it can calculate the new global parameters. much like the
MapReduce algorithm. \cite{googlemapreduce2008} 2) The workers operate
asynchronously constantly updating the parameter server, the parameter server
calculating new global parameters periodically.  \cite{Qirong2013SSP} Whilst
this method is the most common method of machine learning with many benefits,
there are 3 key drawbacks:
\begin{itemize}
    \item The model sacrifices efficiency in either time or computation. Either
    it must wait for all workers to be done each round, or redundant
    computations must be made. \cite{Chilimbi2014ADAM}
    \item when the parameter server is calculating the new global parameters the
    workers are idle or otherwise computing on stale data. \cite{Verbraeken2020MLSurvey}
    \item Each time the parameter server calculates a new global parameters,
    these parameters must be broadcast to each worker simultaneously, consuming
    vast network bandwidth. \cite{LI2014ParameterServers}
\end{itemize}

As has already been addressed frequently models can get so large that they can
no longer feasibly be held within one worker. Therefore there are also
variations with respect to how much of the model each worker operates on: 1) The
model is segmented and split between worker. This is known as \textit{Model
Parallelism} 2) The data is split between the workers which have their own full
local models, but are synced with each other at periodic intervals. This is
known as \textit{Data Parallelism} \cite{Xing2015Petuum} Though model
parallelism shows promise it has its own set of drawbacks:
\begin{itemize}
    \item Often in model parallelism nodes do not communicate with each other,
    this makes it performing algorithms such as Stochastic Gradient Decent
    difficult as clusters of nodes are isolated.
    \item some model parameters take more algorithmic iterations to converge
    than others, so that they converges at the same rate,
    this means that some nodes may be idle, not spreading the load equally or
    efficiently. \cite{Dean2012Distbelief}
    \item Because some parameters converge at different rates, a scheduler is
    needed. However this in turn require more computational overhead and
    communication and reduces iteration throughput. \cite{kim2016STRADS}
\end{itemize}

\subsection{Aims}
My solution to address these issues raised above is to introduce a new model for
Distributed Machine Learning: \textit{Distiller}. Distiller is a Ring
Topological, Model Parallel Distributed Machine Learning framework focusing on
optimising Distributed Stochastic Gradient Descent.
\par

Each of the limitations I have described before I aim to address with the
Distiller Framework:
\begin{itemize}
    \item Distiller aims to reduce the time workers are idle to lower than the
    parameter server model. This will manifest itself by distributing the work
    between workers more proportionally, This means each computers resource is
    used more efficiently and training should be faster per iteration.
    \item The system will also not have a global parameter store, meaning there
    will be no need for communication of weights across the network
    \item As the node network has a ring topology not a star topology due the
    lack of a need of a parameter server prevents one node getting flooded with
    data each iteration. This isn't scalable, as when the workers increase so
    does the data flooding the parameter server leading to bottlenecks. In
    contrast the ring topology allows data to flow in lockstep, with no
    bottlenecking
    \item There will be no isolated nodes as can be an issue in many model
    parallel designs, but they can communicate with each other by design.
    \item The convergence rate will be consistent across all nodes, and they are
    all sharing a global problem rather than working on their own subproblems.
    \item Because the convergence rate is the same across all nodes no scheduler
    is necessary. Instead scheduling is managed in a decentralised manner via
    communication with adjacent nodes.
\end{itemize}






  


% \newline
% holds the global weights of the model. The workers checkout the global
% parameters and use training data to perform a gradient descent algorithm. The
% workers then sends the parameter server the new gradients they have calculated. The
% parameter server aggregates the responses from the workers and then applies it
% to the global weights in the parameter server.










\subsection{Problem}
\todo{Address problems and what are the problems? \\ many solutions\\}
T\lipsum[4]

\subsection{Overview}
This document is split up into the following sections:
\begin{itemize}
 \item \textbf{Section 1} Current section. Introduce the project and its aims.
 \item \textbf{Section 2} Presents related research material and similar applications and areas.
 \item \textbf{Section 3} Gives an overview on the technological choices that will be used.
 \item \textbf{Section 4} Project plan and time management.
 \item \textbf{Section 5} Summary of previous sections.
 %\item \textbf{Section 5} presents the design stages of the application.
 %\item \textbf{Section 6} outlines the application requirements and specifications
 %\item \textbf{Section 7} introduces the implementation and how they have been implemented.
 %\item \textbf{Section 8} describes the testing process
 %\item \textbf{Section 9} presents potential ideas for future work on the application.
\end{itemize}

