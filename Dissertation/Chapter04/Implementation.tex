%-----------------------------------------------------------------------------------------
\clearpage
\section{Implementation}
%-----------------------------------------------------------------------------------------

\subsection{Tooling}
Implementing a distributed neural network is too large a task to be undertaken
from scratch. Therefore its necessary to used existing tools, to make the
development viable in the time given. This is difficult as not many languages
lend themselves to both distributed systems and neural networks.

To ensure high performance the project could be implemented in C++. While C++ is
often very performant and also has low level bindings for ML libraries such as
TensorFlow. However even the creator of the language sees the need to improve
its ability to improve its distributed performance. \cite{stroustrupInterview}

Python has great tooling for neural networking, such as TensorFlow
\cite{abadi2016tensorflow}, and PyTorch \cite{paszke2019pytorch}. Moreover it
has great support for numerical computing with NumPy \cite{harrisNumpy2020}.
These are performant too, by calling C functions or creating code which is
optimised to run on GPUs to parallelise computation. However due to the Global
Interpreter Lock (GIL) python is infamously bad at concurrency, while its
distributed tooling is implemented in native python code, which lack of speed
and could bottleneck the performance gained from using NumPy and TensorFlow.

Ultimately I decided to use Elixir as the programming language of
implementation. This is because Elixir was designed for developing highly
concurrent distributed systems. It does this by having a uniquely brilliant
concurrency model. As opposed to OOP languages where 'everything is an object'
in Elixir 'everything is a process'. This means the default way of writing the
language enables it to be concurrent and scalable. Elixir also has the ability
to communicate with other Elixir programs over the network using its own
application protocol on top of TCP/IP. Meaning its as easy to communicate with
local processes on your own machine as processes on another machine running an
Elixir program. Its also been used by artificial intelligence researchers before
as the process concurrency models effortlessly lends itself to modelling
neurons. \cite{sherNeuroevolutionThroughErlang} Using Elixirs native float and
arithmetic implementation would be slower than a C++ or a NumPy implementation,
luckily there is a stable package which supports matrix calculations even faster
than those in NumPy called Matrex. \cite{matrex}

The only drawback of using Elixir is at the time of development it didn't have a
strong machine learning library, which means implementing the mathematics of the
neural network myself. While this was a sizeable amount of work to do, it had
the benefit that I didn't to wrestle with an opinionated API such as TensorFlow,
I could create my own API to meet my ends.